\section{MySQL Commands}
    For all command examples, we are using the hostname of "jsubmit.jlab.org", the username of "robertej". Substitute these values for your needs. N.b. SQL has a formalism where all commands are capitalized and all items are lower cased. We will try to stick to that in all examples but note that SQL is actually case insenstitive as a language. Also, the language must end each command in a semicolon. \\
    
    
    \subsection{Connecting to Database}
        To establish an interactive session on the command line:
        \begin{lstlisting}
            mysql -h jsubmit.jlab.org -u robertej -p 
        \end{lstlisting}
        
        If you don't want to have an interactive session, or don't want to enter a password each time, you can pass a configuration file as:
        
        \begin{lstlisting}
            mysql --defaults-extra-file=msql_conn.txt     
        \end{lstlisting}
        
        Where mysql\_conn.txt is a text file of the form:
        
        \begin{lstlisting}
            [client]                                                                                            
            user='robertej'                                                                                                   
            password='realpassword'
            host='jsubmit.jlab.org'
            database='CLAS12OCR' 
        \end{lstlisting}
    
    \subsection{Basic MySQL Commands}
    
        See available databases:
        
        \begin{lstlisting}
            SHOW DATABASES;
        \end{lstlisting}
        
        Create a databases:
        
        \begin{lstlisting}
            CREATE DATABASE CLAS12OCR;
        \end{lstlisting}
        
        Delete a databases:
        
        \begin{lstlisting}
            DROP DATABASE CLAS12OCR;
        \end{lstlisting}
        
        Enter a particular database:
        
        \begin{lstlisting}
            USE CLAS12OCR;
        \end{lstlisting}
        
        See available tables in a database:
        
        \begin{lstlisting}
            SHOW TABLES;
        \end{lstlisting}
        
        See schema of a particular table:
        
        \begin{lstlisting}
            DESCRIBE CLAS12OCR.users;
        \end{lstlisting}
        
        Leave the interactive MySQL environment:
        
        \begin{lstlisting}
            EXIT;
        \end{lstlisting}
    
    \subsection{Advanced MySQL Commands}
    
    How to see size of databases (copy verbatim):

        \begin{lstlisting}
            SELECT table_schema "DB Name",
                    ROUND(SUM(data_length + index_length) / 1024 / 1024, 1) "DB Size in MB" 
            FROM information_schema.tables 
            GROUP BY table_schema; 
        \end{lstlisting}
    
    How to dump all information from a database into a text file:
    
    \begin{lstlisting}
        mysqldump -h jsubmit.jlab.org --skip-comments --skip-extended-insert -u robertej -p CLAS12OCR > test10.sql  
    \end{lstlisting}
    
    The two skip options are optional and only include information extraneous to the actual data in the database. They can be omitted without error.\\
    
    How to make a copy of a database to another MySQL DB:

    \begin{lstlisting}
        mysqldump -h jsubmit.jlab.org -u robertej -p CLAS12OCR | mysql -h jsubmit.jlab.org -u robertej -p CLAS12OCRBackup
    \end{lstlisting}    
    
    \subsection{Common MySQL Commands in CLAS12OCR}
    
        Get list of all users and current running jobs:
        
        \begin{lstlisting}
            SELECT user, total_running_jobs FROM users;
        \end{lstlisting}
        
        More usefully, see the 10 most recent submissions on the database:
        
        \begin{lstlisting}
            SELECT user, client_time, run_status FROM submissions ORDER BY user_submission_id DESC LIMIT 10;
        \end{lstlisting}
        
        
        If you want to wrap everything all up in one, you can do: 
        \begin{lstlisting}
        mysql --defaults-extra-file=msql_conn_test.txt -N -s --execute='select user, client_time, run_status from submissions ORDER BY user_submission_id DESC LIMIT 10;' 
        \end{lstlisting}
        
        
        How to copy runscript from the database to a text file:
        \begin{lstlisting}
        mysql --host jsubmit.jlab.org --user='username' --password='password' --database='CLAS12OCR' --execute='SELECT runscript_text FROM Submissions WHERE submissionID = 1; ' > example.txt
        \end{lstlisting}

    \subsection{Permissions in MySQL}
    This section is yet to be populated
    \iffalse
    mysql permissions:
    show users:
    
    select user, host from mysql.db where db='CLAS12OCR';
    
    show grants:
    
    show grants for 'clas12jserver'@'%.jlab.org';
    
    GRANT ALL ON . TO 'clas12jserver'@'%.jlab.org';
    
    current users that access the db:
    
    show processlist;
    
    \fi

\section{Testing with SQLite}
    \subsection*{Create SQLite DB}
        \begin{lstlisting}
            python utils/create_database.py --lite=name.sql
        \end{lstlisting}
    
    \subsection*{Submit job on client side to SQLite DB}
        \begin{lstlisting}
            python client/src/SubMit.py --lite=name.sql -u=username client/scard_type1.txt
        \end{lstlisting}
        
        
         sqlite run to download the running script and the gcard. Assuming DB is in the same dir
    \begin{lstlisting}
        sqlite3 CLAS12_OCRDB.db "SELECT runscript_text FROM
        Submissions WHERE submissionID = $submissionID"  > $nodeScript
    \end{lstlisting}
    
\section{Statistics and Logging}
    \subsection*{GEMC JSON Web Interface Statistics}
         To test on sqlite that things work as intended: 
         \begin{lstlisting}
            python3 utils/gemc_json_logging.py -lite=CLAS12OCR.db -t -o testoutput.log
        \end{lstlisting}
        ~\\
        \newline
        Normal use (with no arguments, default will use mysql db and output to osgLog.json)
        \begin{lstlisting}
            python3 utils/gemc_json_logging.py 
        \end{lstlisting}


\section{Using Travis-CI}
    \href{https://travis-ci.org/}{Travis Location}
    
    Important parts of Travis CI:\\
    There are currently documents important for Travis CI:\\
    
    .travis.yml\\
    requirements.txt\\
    travis.sql\\
    travis\_tests.py\\
    readtests.py\\
    
    \subsection{.travis.yml}
        
        \begin{lstlisting}
        
            language: python
            python:
              - "2.7"
            #  - "3.6"      # current default Python on Travis CI
            # command to install dependencies
            services:
             - mysql
            before_install:
             - mysql -u root --password="" < test/travis.sql
            # - echo "USE mysql;\nUPDATE user SET password=PASSWORD('devpassword') WHERE user='dev';\n" | mysql -u root
            install:
            #  - if [[ $TRAVIS_PYTHON_VERSION == 2.7 ]]; then pip install -r req-py2.txt; fi
            # - pip install -r req-py2.txt; python_version == "2.7"
             - pip install -r test/requirements.txt
            # - pip install mysql-python
            # command to run tests
            
            #script: python2 testapp.py || python3 testapp.py #For now, can't get python3 to work properly
            script: python2 test/travis_tests.py
        \end{lstlisting}
    
    This file must be named as ".travis.yml" and must be found in the main repository directory. It is the launching point for the CI tests. Here you can specify which python versions you want to test, what services to install, and which programs to run.
    
    \subsection{requirements.txt}
        This file contains which python modules need to be installed in the virtual environment. If something is missing, travis will throw an error, so its not so difficult to get right.
        
    \subsection{travis.sql}
                \begin{lstlisting}
                    # Create DB
                    CREATE DATABASE IF NOT EXISTS `CLAS12OCR`;
                    CREATE DATABASE IF NOT EXISTS `CLAS12TEST`;
                \end{lstlisting}
                This initializes the MySQL databases
                
    \subsection{travis\_tests.py}
        This creates all the logically structure and is a framework to execute commands. Its not so important to look at unless something is broken. 
        
    \subsection{readtests.py}
        This is the actual location where what commands to be submitted on Travis-CI are actually specified. The current command structure is:
        
                \begin{lstlisting}
                 verify_submission_success = command_class('Verify scard submission success',
                                    ['sqlite3','utils/CLAS12OCR.db','SELECT user FROM submissions WHERE user_submission_id=1'],
                                    'robertej\n')
                \end{lstlisting}        
        
       The structure is a class where the first element is the desired name of the command (user defined), the second is the actual command to execute on the terminal, broken up between spaces, and the third element is what the expected output is from the command. The default is "0" if no output comparison is desired to be made. Travis will fail if a non-zero output is specified and is not observed, even if the command processes successfully. 
    
    




